\documentclass{article}

\usepackage[mathtabular,
            minionint,
            mathlf]%
            {MinionPro}
\usepackage{booktabs}
\usepackage[table]{xcolor}
\usepackage{geometry}
\usepackage[fontsize=13pt]{scrextend} 
\usepackage{dirtree}
\usepackage{hyperref}
\usepackage[capitalise]{cleveref}
\usepackage{pgfplotstable}
\usepackage{enumitem}

\pagenumbering{gobble}
\definecolor{link}{HTML}{8B0000}
\definecolor{purple}{HTML}{660066}

\hypersetup{%
  hidelinks,
  colorlinks=true,
  allcolors=link
}

\setlength\parindent{0pt}

% Renew table environment to always centre content and use Minion Pro's tabular
% lining figures
\let\originaltable\table
\let\endoriginaltable\endtable
\renewenvironment{table}[1][ht]{%
	\originaltable[#1]
	\centering
	\figureversion{tabular,lf}
	}%
	{\endoriginaltable}

\newcommand{\tb}{\textbackslash}

\begin{document}

\begin{center}
  \Large\textsc{Exercise 1}
\end{center}

Let's try use some of the things we learnt in the last section. If you
get stuck at any point put your hand up and I'll come over to help.\\

Create a folder titled \texttt{sage\_latex} on your computer. Download
\href{https://jwalton.info/assets/teaching/latex/scarlet.tex}{this file} 
and save it in the folder which you just created.

\begin{enumerate}
  \item Check that you can compile the file you just downloaded
    \texttt{scarlet.tex}.

  \item Extract the preamble from \texttt{scarlet.tex} and save it in its own
    separate file. You should name this separate file
    \texttt{preamble.tex}.

  \item Use the \texttt{\tb include} command to insert
    \texttt{preamble.tex} into \texttt{scarlet.tex}. Ensure that you can
    still compile \texttt{scarlet.tex}.

  \item{Create a new folder called \texttt{parts}. Extract the parts from 
    \texttt{scarlet.tex} and put them in their own separate files within
    the \texttt{parts} folder. Your folder structure should now look
    like the following:
    \vspace{0.25cm}
    \dirtree{%
    .1 \hspace{-1cm} sage\_latex/. 
    .2 scarlet.tex. 
      .2 parts/. 
      .3 part\_1.tex. 
      .3 part\_2.tex. 
      .2 preamble.tex. 
    }
    Ensure that you can still compile \texttt{scarlet.tex} with this new 
    folder structure.}

  \item Create a folder \texttt{fig/} within \texttt{sage\_latex/}. Download a
    detective-related image (think pipes, magnifying glasses etc.) and
    save it in the \texttt{fig/} folder.

  \item Add the \texttt{fig/} folder to \texttt{graphicx}'s
    \texttt{graphicspath}.

  \item Include your detective image on the book cover. Your call to
	  \texttt{\tb includegraphics} should \emph{not} reference the
    \texttt{fig/} folder.
    
  \item If you got this far consider creating a directory structure and layout
    to use for your thesis, downloading the MSP template or the classic
    thesis style.
\end{enumerate}

\end{document}
