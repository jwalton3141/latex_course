\documentclass{article}

\usepackage[mathtabular,
            minionint,
            mathlf]%
            {MinionPro}
\usepackage{booktabs}
\usepackage[table]{xcolor}
\usepackage{geometry}
\usepackage[fontsize=13pt]{scrextend} 
\usepackage{dirtree}
\usepackage{hyperref}
\usepackage[capitalise, nameinlink]{cleveref}
\usepackage{pgfplotstable}
\usepackage{enumitem}

\pagenumbering{gobble}
\definecolor{link}{HTML}{8B0000}
\definecolor{purple}{HTML}{660066}

\hypersetup{%
  hidelinks,
  colorlinks=true,
  allcolors=link
}

\setlength\parindent{0pt}

% Renew table environment to always centre content and use Minion Pro's tabular
% lining figures
\let\originaltable\table
\let\endoriginaltable\endtable
\renewenvironment{table}[1][ht]{%
	\originaltable[#1]
	\centering
	\figureversion{tabular,lf}
	}%
	{\endoriginaltable}

\newcommand{\tb}{\textbackslash}


\begin{document}

\begin{center}
  \Large\color{black!80}\textsc{Exercise 2}
\end{center}

Let's try use some of the things we learnt in the last section. If you
get stuck at any point put your hand up and I'll come over to help.

\begin{enumerate}[label=\color{purple}\theenumi.]

  \item Within the \texttt{sage\_latex} folder create the file
    \texttt{macros.tex}.

  \item Give \texttt{macros.tex} a \texttt{documentclass}.
  Ensure you can compile \texttt{macros.tex} with some dummy text in the document
  body.

  \item Create a macro \texttt{\tb newt} which typesets the 
  equation $F = m \ddot{x}$. \textsl{Hint:} In math-mode two dots can be typeset
  above a letter with \texttt{\tb ddot}.

  \item Add the line ``\texttt{From Newton's second law we have that
  \tb newt.}'' to the body of \texttt{macros.tex}. Ensure that 
  \texttt{macros.tex} compiles.

  \item Create a macro \texttt{\tb triang} which takes one argument.
  This macro should typeset the summation $\Sigma_{i=1}^N i$ where $N$ is 
  specified by a parameter passed to \texttt{\tb triang}.

  \item Add the line ``\texttt{The triangular numbers
  \tb triang\{3\}=6, \tb triang\{7\}=28 
  and \tb triang\{31\}=496 are also examples of perfect numbers.}'' 
  to \texttt{macros.tex}. Ensure you can compile \texttt{macros.tex}.
  
  \item Create a macro \texttt{\tb gauss} to typeset the 
  integral: $\int_{-\infty}^{\infty} e^{-x^2} \textrm{d}x$. Change
  \texttt{\tb gauss} so that it takes a parameter which sets the
  symmetric limits of the integral. Finally, alter \texttt{\tb gauss}
  so that its default parameter is \texttt{\tb infty}.

  \item Add the line ``\texttt{The Gaussian integral, also known as the
  Euler-Poisson integral, evaluates to
  \tb gauss=\$\tb sqrt\{\tb pi\}\$. For
  a complete proof \\ of this result we shall consider \$I(a)\$=\tb gauss[a]
  and the limit \\\$\tb lim\_\{a\tb rightarrow\tb infty\} I(a)\$}''. Again,
  ensure that you can compile\\ \texttt{macros.tex} with this line included.
 
\end{enumerate}

\end{document}
