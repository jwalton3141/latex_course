
\section{Managing large documents}

\begin{frame}[fragile]{Modular \LaTeX}
  \begin{itemize}
	  \item For smaller projects it is \emph{okay} to keep everything in a single
      \lstinline|.tex| file
    \item For more involved projects (your thesis) this approach \emph{quickly}
      becomes cumbersome
    \item The \lstinline|\include| command makes it possible to
      break your document down into smaller chunks
    \item Working with smaller chunks is more manageable 
  \end{itemize}
\end{frame}

\begin{frame}{Structure}
  An example structure for a thesis project could look like the following:
  \\\vspace{0.5cm}
  \dirtree{%
    .1 thesis/. 
    .2 thesis.tex. 
    .2 chapters/. 
    .3 chapter\_1.tex. 
    .3 chapter\_2.tex. 
    .3 chapter\_3.tex. 
    .2 internal/. 
    .3 preamble.tex. 
    .2 fig/.
    .3 science.png.
    .2 references.bib. 
  }
\end{frame}

\begin{frame}[fragile]{Example thesis.tex}
  \begin{lstlisting}
\documentclass[12pt]{report}

\usepackage[mathtabular,
            minionint,
            mathlf]%
            {MinionPro}
\usepackage{booktabs}
\usepackage[table]{xcolor}
\usepackage{geometry}
\usepackage[fontsize=13pt]{scrextend} 
\usepackage{dirtree}
\usepackage{hyperref}
\usepackage[capitalise]{cleveref}
\usepackage{pgfplotstable}
\usepackage{enumitem}

\pagenumbering{gobble}
\definecolor{link}{HTML}{8B0000}
\definecolor{purple}{HTML}{660066}

\hypersetup{%
  hidelinks,
  colorlinks=true,
  allcolors=link
}

\setlength\parindent{0pt}

% Renew table environment to always centre content and use Minion Pro's tabular
% lining figures
\let\originaltable\table
\let\endoriginaltable\endtable
\renewenvironment{table}[1][ht]{%
	\originaltable[#1]
	\centering
	\figureversion{tabular,lf}
	}%
	{\endoriginaltable}

\newcommand{\tb}{\textbackslash}
 

\begin{document}

\include{chapters/chapter_1}
\include{chapters/chapter_2}
\include{chapters/chapter_3}

\bibliography{references}

\end{document}
  \end{lstlisting}
\end{frame}

\begin{frame}[fragile]{Example internal/preamble.tex}
  \begin{lstlisting}
% Preamble, packages, commands etc.
\usepackage{microtype}
\usepackage{booktabs}
\usepackage{cleveref}
\usepackage{graphicx}
  \end{lstlisting}
  \vspace{-0.2cm}\vdots
  \begin{lstlisting}
% Make it easier to include figures 
\graphicspath{{fig/}}
  \end{lstlisting}
\end{frame}

\begin{frame}[fragile]{Example chapters/chapter\_1.tex}
  \begin{lstlisting}
\chapter{Literature review}
\label{cha:lit_review}

Here's stuff others did which I don't really
understand\ldots
  \end{lstlisting}
\end{frame}

\begin{frame}[fragile]{Compile a single chapter}
  \lstinline|\includeonly| allows the compilation of a single chapter,
  without messing up references, page numbers etc.
  \begin{lstlisting}
\documentclass[12pt]{report}

\usepackage[mathtabular,
            minionint,
            mathlf]%
            {MinionPro}
\usepackage{booktabs}
\usepackage[table]{xcolor}
\usepackage{geometry}
\usepackage[fontsize=13pt]{scrextend} 
\usepackage{dirtree}
\usepackage{hyperref}
\usepackage[capitalise]{cleveref}
\usepackage{pgfplotstable}
\usepackage{enumitem}

\pagenumbering{gobble}
\definecolor{link}{HTML}{8B0000}
\definecolor{purple}{HTML}{660066}

\hypersetup{%
  hidelinks,
  colorlinks=true,
  allcolors=link
}

\setlength\parindent{0pt}

% Renew table environment to always centre content and use Minion Pro's tabular
% lining figures
\let\originaltable\table
\let\endoriginaltable\endtable
\renewenvironment{table}[1][ht]{%
	\originaltable[#1]
	\centering
	\figureversion{tabular,lf}
	}%
	{\endoriginaltable}

\newcommand{\tb}{\textbackslash}

\includeonly{chapters/chapter_2}

\begin{document}

\include{chapters/chapter_1}
\include{chapters/chapter_2}
\include{chapters/chapter_3}
  \end{lstlisting}
  \vspace{-0.2cm}\vdots
\end{frame}

\begin{frame}{masthesis.sty}
  \begin{itemize}
    \item \href{https://wiki.mas.ncl.ac.uk/mas/ThesisTemplate?highlight=\%28masthesis\%29}%
      {A thesis template for MSP students}
    \item The template is modular and has a structure \emph{similar} to the one
      given above
    \item For non-MSP students, or those who would like a different style, 
      \href{https://ctan.org/pkg/classicthesis?lang=en}%
      {the `classic thesis' style} is a good option
  \end{itemize}
\end{frame}

\begin{frame}{Version control}
  \begin{itemize}
    \item Version control allows you to track and manage changes in code, and
      collaborate with others
    \item I'd recommend using version control to manage your thesis
    \item Plug: a colleague and I are teaching an upcoming PGRDP workshop
      \href{https://workshops.ncl.ac.uk/view/book/modal/43153/}%
      {Introduction to Git and GitHub}
  \end{itemize}
\end{frame}


\begin{frame}[fragile]{Spell checking}
  Spell checking \texttt{.tex} files is complicated by latex commands.

  For those comfortable working at the command line I'd recommend
  \href{http://aspell.net/}{aspell} 
  (or \href{https://www.gnu.org/software/ispell/}{ispell}
  or \href{http://hunspell.github.io/}{hunspell}).

  Interactive spell-check:
  \begin{lstlisting}
$ aspell -t -c chapters/chapter1.tex
  \end{lstlisting}
  Non interactive spell-check (lists mistakes):
  \begin{lstlisting}
$ cat chapters/chapter1.tex | aspell list -t
  \end{lstlisting}%
  Custom dictionary and commands to ignore can be added with
  \lstinline|--add-extra-dicts| and \lstinline|--conf| respectively
\end{frame}

\begin{frame}[fragile]{Spell checking}
  Some IDEs have inbuilt spell checkers:
  \begin{itemize}
    \item \href{http://www.xm1math.net/texmaker/}{Texmaker}
      (checks contents of commands still)
    \item \href{https://www.texstudio.org/}{Texstudio} (seems to have the best
       spellchecker)
  \end{itemize}
  More generally:
  \href{https://tex.stackexchange.com/questions/339/latex-editors-ides}%
  {here is a list of editors} and their features
\end{frame}

\begin{frame}[fragile]{Word count}
  For final submission (it will creep up on you, I promise) you \textsl{need}
  to submit a word count.

  Counting words in a .tex file is again complicated by the presence of latex
  commands.

  For command line users I'd recommend trying
  \href{https://www.freebsd.org/cgi/man.cgi?query=detex}{detex} and 
  \href{http://pubs.opengroup.org/onlinepubs/9699919799/utilities/wc.html}{wc}:
  \begin{lstlisting}
$ detex -le equation,table thesis.tex | wc -w
  \end{lstlisting}
\end{frame}

\begin{frame}{Word count}
  \begin{itemize}
    \item \href{http://app.uio.no/ifi/texcount/online.php}{Online tool}
      (chapters counted one at at time)
    \item \href{http://www.xm1math.net/texmaker/}{Texmaker}'s integrated pdf
       viewer has wordcount (right click pdf)
    \item \href{https://www.texstudio.org/}{Texstudio} (tools $\rightarrow$
      analyze text; chapters one at a time)
  \end{itemize}
\end{frame}

\begin{frame}[standout]
  \href{https://jwalton.info/assets/teaching/latex/exercise_1.pdf}%
  {\color{white}Exercise 1}
\end{frame}
