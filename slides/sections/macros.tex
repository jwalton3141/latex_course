\section{Custom commands}

\begin{frame}[fragile]{Simple macros}
  Used to simplify repetitive and/or complex formatting.

  Usually specified in the preamble
  \begin{lstlisting}
\newcommand{\name}{definition}
  \end{lstlisting}
\end{frame}

\begin{frame}[fragile]{Simple macros: an example}
  \begin{lstlisting}
\newcommand{\R}{\mathbb{R}}

The set of real numbers are usually represented
by a blackboard captital r: $\R$.
  \end{lstlisting}

  The set of real numbers are usually represented by a blackboard capital r:
  $\R$.
\end{frame}

\begin{frame}[fragile]{Macros with parameters}
  Macros can also be constructed to accept parameters:
  \begin{lstlisting}

\newcommand{\name}[# params]{definition}
  \end{lstlisting}
\end{frame}

\begin{frame}[fragile]{Macros with parameters: an example}
  \begin{lstlisting}
\newcommand{\bb}[1]{\mathbb{#1}}

Other numerical systems have similar notations. 
The complex numbers $\bb{C}$, the rational 
numbers $\bb{Q}$ and the integer numbers
$\bb{Z}$.

  \end{lstlisting}
  Other numerical systems have similar notations. The complex numbers $\bb{C}$,
  the rational numbers $\bb{Q}$ and the integer numbers $\bb{Z}$.
\end{frame}

\begin{frame}[fragile]{Macros with default parameters}
  It is also possible to define macros which take default parameters:
  \begin{lstlisting}
\newcommand{\name}[# params][default #1]{def.}

  \end{lstlisting}
\end{frame}

\begin{frame}[fragile]{Macros with default parameters}
  \begin{lstlisting}
\newcommand{\plusbinomial}[3][2]{(#2 + #3)^#1}

We make a new command to save time writing
expressions of the form $\plusbinomial{x}{y}$
and $\plusbinomial[4]{a}{b}$.

  \end{lstlisting}
  We make a new command to save time writing expressions of the form
  $\plusbinomial{x}{y}$ and $\plusbinomial[4]{a}{b}$.
\end{frame}

\begin{frame}[standout]
  \href{https://jwalton.info/assets/teaching/latex/exercise_2.pdf}%
  {\color{white}Exercise 2}
\end{frame}

